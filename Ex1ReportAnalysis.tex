
\documentclass{article}
%%%%%%%%%%%%%%%%%%%%%%%%%%%%%%%%%%%%%%%%%%%%%%%%%%%%%%%%%%%%%%%%%%%%%%%%%%%%%%%%%%%%%%%%%%%%%%%%%%%%%%%%%%%%%%%%%%%%%%%%%%%%%%%%%%%%%%%%%%%%%%%%%%%%%%%%%%%%%%%%%%%%%%%%%%%%%%%%%%%%%%%%%%%%%%%%%%%%%%%%%%%%%%%%%%%%%%%%%%%%%%%%%%%%%%%%%%%%%%%%%%%%%%%%%%%%
\usepackage{amsmath}

\setcounter{MaxMatrixCols}{10}
%TCIDATA{OutputFilter=LATEX.DLL}
%TCIDATA{Version=5.50.0.2953}
%TCIDATA{<META NAME="SaveForMode" CONTENT="1">}
%TCIDATA{BibliographyScheme=Manual}
%TCIDATA{Created=Thursday, February 15, 2018 19:16:23}
%TCIDATA{LastRevised=Thursday, February 15, 2018 20:31:57}
%TCIDATA{<META NAME="GraphicsSave" CONTENT="32">}
%TCIDATA{<META NAME="DocumentShell" CONTENT="Standard LaTeX\Blank - Standard LaTeX Article">}
%TCIDATA{Language=American English}
%TCIDATA{CSTFile=40 LaTeX article.cst}

\newtheorem{theorem}{Theorem}
\newtheorem{acknowledgement}[theorem]{Acknowledgement}
\newtheorem{algorithm}[theorem]{Algorithm}
\newtheorem{axiom}[theorem]{Axiom}
\newtheorem{case}[theorem]{Case}
\newtheorem{claim}[theorem]{Claim}
\newtheorem{conclusion}[theorem]{Conclusion}
\newtheorem{condition}[theorem]{Condition}
\newtheorem{conjecture}[theorem]{Conjecture}
\newtheorem{corollary}[theorem]{Corollary}
\newtheorem{criterion}[theorem]{Criterion}
\newtheorem{definition}[theorem]{Definition}
\newtheorem{example}[theorem]{Example}
\newtheorem{exercise}[theorem]{Exercise}
\newtheorem{lemma}[theorem]{Lemma}
\newtheorem{notation}[theorem]{Notation}
\newtheorem{problem}[theorem]{Problem}
\newtheorem{proposition}[theorem]{Proposition}
\newtheorem{remark}[theorem]{Remark}
\newtheorem{solution}[theorem]{Solution}
\newtheorem{summary}[theorem]{Summary}
\newenvironment{proof}[1][Proof]{\noindent\textbf{#1.} }{\ \rule{0.5em}{0.5em}}
\input{tcilatex}
\begin{document}


\begin{center}
{\LARGE REPORT \ }\emph{\ \ }\ \ \ \ \ \ \ 
\end{center}

\textit{\ In this report we want to analyze the method we used in order to
regress two random variables which are correlated with a specific
correlation value and to present the results of the regression. To
accomplish this analysis we used Matlab programming\ \ \ \ \ \ \ \ \ \ \ \ \
\ \ \ \ \ \ \ \ \ \ \ \ \ \ \ \ \ \ \ \ \ \ \ \ \ \ \ \ \ \ \ \ \ \ \ \ \ \
\ \ \ \ \ \ \ }

\section{\textit{Methodology of generating two random variables with a
specific correlation \ \ \ \ \ }}

\textit{We created two random variables x,\ y which are correlated at 0.1(or
10\%). In more details, we generated a random matrix \ }$M\ $\textit{\ with
2 columns, each column corresponds to the vector we want to derive (the
elements of this matrix are drawn from a standard normal distribution).
Furthermore, we multiplied the matrix with standard deviation and we add the
mean, where variance ( square root of the variance is the standard
deviation) and mean drawn from a normal distribution. Afterwards, we
multiplied M with the upper triangular matrix }$L$\textit{\ which is
obtained using the Cholesky decomposition of the desired correlation matrix }%
$R$\textit{.}

\section{\textit{Regression Analysis}}

\subsection{\textit{Model}}

$y=b_{0}+b_{1}\ast x+\in $\textit{, \ where }$\in $\textit{\ is the error
term.}

\subsection{\textit{Regression Output}}

$\overset{\wedge }{y}\mathit{=}\underset{(0.0307)}{-0.0077}\mathit{+}%
\underset{(0.0308)}{+0.0753}\mathit{\ast x}$

\textit{Moreover, we can observe from our results, that the p-value of the }$%
b_{1}$\textit{, is equal to 0.0148 which implies that this beta coefficient
is statistical significant for }$a=0.05$\textit{. Whereas, the p-value of }$%
b_{0}$\textit{\ is equal to 0.8014 so is statistical insignificant. Finally
the degrees of freedom are 998.}

\section{Summary}

\textit{We can conclude that the estimated correlation coefficient between x
and y is 0.0771 which is smaller compared  to the real correlation
coefficient. }

\end{document}
